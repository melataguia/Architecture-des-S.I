\documentclass[12pt,a4paper]{article}
\usepackage[utf8]{inputenc}
\usepackage[T1]{fontenc}
\usepackage[utf8]{inputenc}
\usepackage[french]{babel}
\usepackage{graphicx}
\usepackage[T1]{fontenc}
\begin{document}
\section*{Introduction}
  Youtube  est un site web d'hébergement de vidéos et média social sur lequel les utilisateurs ont la possibilité d'envoyer, regarder, 
commenter, évaluer et partager des vidéos en streaming.le machine learning est un domaine d'investigation consacré à la compréhension et 
à la construction de méthodes qui
 «apprennent», c'est-à-dire des méthodes qui exploitent les données pour améliorer les performances sur un ensemble de tâches. Ainsi,Les 
révolutions scientifiques sont toujours inspirées de la nature humaine, qu'elles soient physique ou intellectuelle. Ainsi 
quelle place occupe le Machine Learning dans le système de recommandation de YouTube? Et comment fonctionne le système de recommandation
 de Youtube? Repondre a ces questions fera l'objet de notre étude. 
\section{Approche base sur le contenu}	
 une approche basé sur le contenu consiste a repérer les éléments qui correspondent le mieux avec les préférences de l'utilisateur. Cette technique se focalise sur les caractéristiques des produits afin de recommander aux utilisateurs les produits qui auront de nouvelles propriétés 
similaire avec des produits avec lesquels ils ont déjà interagit. 
\section{Approche basée sur le filtrage collaboratif }	
Une approche basée sur le filtrage collaboratif produisent des recommandations en calculant la similarité entre les préférences d'un utilisateur et d'autres utilisateurs. Cette méthode s'appuie sur l'information relative aux utilisateurs pour calculer la similarité entre deux produits utilisateurs.
\section{Approche hybride}
L'approche hybride utilisent de différents types d'approche de recommandation ou s'appuie sur la leurs logique. Par exemple un tel système peut utiliser a la fois des connaissances extérieures et les caractéristiques des éléments combinant ainsi des approches collaboratives basées sur le contenu.	
\section{Système de recommandation d'après notre recherche}
Ici ,En fonction de notre recherche , YouTube peut nous recommander des vidéos similaire a notre
recherche. Alors ce qui va se passer dans ce cas c'est que l'algorithme va utiliser les différents termes(expressions) de notre recherche et aller
dans la base de données rechercher les éléments similaires. Et lorsque nous Postons une vidéo sous youTube nous mettons les différents mots 
susceptibles de retrouver notre vidéos 

\end{document}
